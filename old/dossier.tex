\documentclass[11pt]{ritsi/article}

\usepackage{ritsi/frontpage}
\usepackage{eurosym}

\title{IV RITSI Congress}
\author{Sponsorship dossier}

\begin{document}

\maketitle

\tableofcontents

\cleardoublepage

\section{Introducción}

El próximo 15 de noviembre la Universidad del País Vasco / Euskal Herriko Unibertsitatea será la anfitriona del IV Congreso Estatal de RITSI (Reunión Estudiantes de Ingenierías Técnicas y Superiores en Informática) en su campus de Gipuzkoa. Durante esta jornada los asistentes procedentes de puntos de todo el país tendrán la oportunidad de conocer de primera mano a empresas y organismos del sector, pudiendo entrar en contacto con sus últimos proyectos y tecnologías.

San Sebastián, uno de los motores de desarrollo más activos de España, será la ciudad encargada de acoger el evento. Siendo una pieza clave del desarrollo tecnológico y contando con entidades como el Parque Científico Tecnológico de Gipuzkoa, el Centro de Estudios e Investigaciones Técnicas o la propia Universidad, constituye un enclave idóneo para celebrar un congreso enfocado al mundo en constante evolución y reinvención de la Informática.

Donostia es también uno de los principales núcleos turísticos y culturales del país. Prueba de esto es la titularidad como capital europea de la cultura en 2016, otorgada gracias a un trabajo constante y una apuesta segura en este ámbito. El Museo de San Telmo, la Playa de la Concha, el Palacio de Miramar o la Catedral del Buen Pastor son tan solo algunos ejemplos de las maravillas naturales y arquitectónicas de las que podemos disfrutar en la capital gipuzkoana.

No debemos obviar tampoco los múltiples festivales y ferias que encontramos a lo largo del año, como el Festival Internacional de Cine de San Sebastián, el Concurso y Festival Internacional de Fuegos Artificiales, el festival de Jazz o los conciertos del Orfeón Donostiarra.
 
\section{¿Qué es RITSI?}
La Reunión de Estudiantes de Ingenierías Técnicas y Superiores en Informática (RITSI) es la asociación a nivel estatal encargada de representar a todo el alumnado de las titulaciones de Ingeniería Informática. Fundada en 1992, forman parte de ella toda aquella Delegación y Consejo de Estudiantes de las facultades y escuelas universitarias que imparten estudios en cualquier rama de la Ingeniería Informática. Sus objetivos principales son la la representación de los más de 60000 estudiantes de Ingenierías Informáticas, defendiendo sus derechos e intereses, actuar como nexo entre el alumnado y organismos del sector como Colegios Profesionales o instituciones privadas y trabajar por el futuro profesional de estos estudiantes, además de la realización actividades de divulgación de interés para el colectivo.

La dinámica de trabajo de la Asociación es asamblearia, reuniéndose esta semestralmente en una de las universidades que colaboran con RITSI. El trabajo interasambleario se distribuye en comisiones temáticas y zonales.

La Junta Directiva de RITSI se compone de 5 cargos, actualmente desempeñados por:
\begin{itemize}
    \item Francisco Maestre Torreblanca, de la Universidad de Cádiz, como Presidente.
    \item Alén Blanco Domínguez, de la Universidad de Granada, como Vicepresidente.
    \item Enrique Delgado Rodríguez, de la Universidad de Córdoba, como Secretario.
    \item Rodrigo Alonso Iglesias, de la Universidad de Valladolid, como Tesorero.
    \item Tania Yepes Barbero, de la Universidad de Murcia, como Vocal.
\end{itemize}
 
\section{Origen y fundamento de RITSI}

En la actualidad existen en España aproximadamente 60.000 alumnos matriculados en titulaciones de Ingenierías en Informática.

En cada una de las Facultades o Escuelas donde se imparten estas titulaciones existe una Delegación de Estudiantes que se encarga de representar a los estudiantes en el marco de su Centro y en el de su Universidad. RITSI fue creada en el año 1992 para coordinar las labores de actuación de estas delegaciones, quienes hasta entonces no contaban con ningún organismo que las uniera. Tiene como meta representar a todos los estudiantes de Ingeniería Informática del Estado, defender sus derechos y ser interlocutor válido con los diferentes agentes y asociaciones que tienen relación con nuestros estudios y con nuestra profesión.

La Reunión de Estudiantes de Ingenierías Técnicas y Superiores en Informática es, por tanto, el único organismo formado por estudiantes, con capacidad de coordinar acciones en el ámbito de la ingeniería informática a nivel nacional, y la potestad de representar a los estudiantes de ingeniería informática del país ante los órganos que regulan y rigen nuestra profesión.

\section{Universidad del País Vasco / Euskal Herriko Unibertsitatea}

La Universidad del País Vasco / Euskal Herriko Unibertsitatea, fundada en 1980 y distinguida con la categoría de Campus de Excelencia Internacional, cuenta con 31 centros distribuidos a lo largo de toda la geografía vasca y más de 44000 estudiantes en casi la totalidad de las ramas del conocimiento, constituyendo uno de los pilares del desarrollo en la región, actuando como motor de progreso y de debate intelectual, garantizando una formación calidad para los profesionales del mañana.

La Universidad ofrece entre sus planes de estudios las titulaciones de Grado en Ingeniería en Informática, Grado en Ingeniería Informática de Gestión y Sistemas de Información e Ingeniería Informática, así como una amplia oferta de másteres y doctorados, contando actualmente con más de 1000 estudiantes en esta rama del saber.

\section{¿Qué es el Congreso Estatal RITSI?}
El Congreso es una actividad de RITSI que proporciona a los estudiantes de Ingenierías Informáticas la oportunidad de entrar en contacto con empresas, asociaciones, fundaciones o instituciones nacionales e internacionales de interés en el ámbito informático, a fin de proporcionar un punto de vista sobre el sector que no es posible conocer en el aula. Es también una medio de gran efectividad para dar a conocer a los asistentes el panorama laboral que se encontrarán al terminar sus estudios, poniéndoles frente a frente con profesionales del sector, pudiendo conocer de primera mano su experiencia del mundo laboral, los proyectos en los que actualmente trabajan, o el papel que desempeña su empresa o institución en el sector actualmente.

\section{Antecedentes}

La edición de San Sebastián cuenta con la experiencia de tres antecedentes en los últimos dos años: Alcalá de Henares, Albacete y Salamanca, en orden cronológico. Cada edición ha conseguido superar en asistentes a la anterior, siendo patente el interés que despierta este evento en los estudiantes de Ingeniería Informática y en los organismos del sector que participan en la actividad.

\subsection{I Congreso Estatal RITSI}

La primera edición del Congreso Estatal de Estudiantes de Ingeniería en Informática tuvo lugar en la Universidad de Alcalá en noviembre de 2011. El encuentro fue positivo para los asistentes en dos sentidos: por un lado, se trató una temática de tendencia actual en nuestro país, las metodologías ágiles y el cloud computing, enfocándose desde el punto de vista de su utilización en la empresa, y por otro, se dieron a conocer las oportunidades de empleo en el sector a través de las compañías participantes: Microsoft, Indra, Babel, Cisco, Kybele Consulting y Tuenti.

La acogida superó las expectativas de la organización, congregando en el salón de actos del Edificio Politécnico del Campus Científico-Tecnológico a cerca de 600 asistentes, procedentes tanto de las universidades de la Comunidad de Madrid como de las universidades de Alicante, Córdoba y Salamanca. Este evento fue organizado conjuntamente por RITSI y la Delegación de Estudiantes de la Escuela Técnica Superior en Ingeniería Informática (ETSII) de la Universidad de Alcalá (UAH), en colaboración con la propia UAH, la ETSII y el Consejo de Estudiantes de la UAH, con la colaboración del Ministerio de Educación y Cultura.

La organización y los ponentes manifestaron el éxito del proyecto y comunicaron su deseo de volver a realizarlo en un futuro.

\subsection{II Congreso Estatal RITSI}

La segunda edición del Congreso Estatal RITSI de Estudiantes de Ingeniería en Informática se celebró en el Paraninfo Universitario de Albacete de la Universidad de Castilla-La Mancha en noviembre de 2012, motivada por el buen devenir de la primera edición. En esta ocasión, las temáticas tratadas fueron la "Importancia de la Informática en la Sociedad" y el "Desarrollo de aplicaciones móviles". Se contó con la presencia de representantes de distintos sectores como el bancario, el sanitario o las administraciones públicas, y con empresas referentes en el sector como Research in Motion (RIM), Mozilla y Microsoft, que tomaron parte en este encuentro multitudinario para hablar de desarrollo en plataformas móviles. Asimismo, las empresas participantes mostraron a los asistentes sus oportunidades de empleo y programas formativos para estudiantes. Se contó con stands promocionales de Microsoft y la Fundación Universidad-Empresa, también patrocinadora del evento, para ofrecer sus últimas novedades, productos y servicios.

La realización de este II Congreso fue posible gracias a la conjunta organización de RITSI y la Delegación de Alumnos de la Escuela Superior de Ingeniería en Informática de Albacete, en colaboración con la Universidad de Castilla-La Mancha (UCLM), la Escuela Superior de Ingeniería en Informática de la UCLM y el Consejo de Representantes de Estudiantes de la UCLM. Una vez más, las expectativas de RITSI fueron superadas tras completar un aforo de más de 700 personas con asistentes procedentes de los Campus de Albacete y Ciudad Real de la UCLM, así como de las universidades de Alcalá, Alicante, Autónoma de Madrid, Córdoba, Murcia y Politécnica de Valencia.

\subsection{III Congreso Estatal RITSI}

La tercera edición del Congreso Estatal de RITSI tuvo lugar en el Palacio de Congresos y Exposiciones de Salamanca en marzo de 2013, reuniendo a más de novecientos estudiantes y profesionales de la Ingeniería Informática. En esta ocasión se tomó como hilo conductor la casuística de éxito de grandes empresas del sector que, gracias a una combinación de talento e innovación, han logrado superarse a sí mismas y convertirse en referentes para la industria y la sociedad. Microsoft, Blackberry y Babel, empresas que ya habían colaborado en las ediciones anteriores, participaron de nuevo en el evento. Nuevas colaboradores fueron Amazon, Micro Focus, Anyhelp y Facebook. A su vez, Micro Focus y Facebook ofrecieron talleres en grupos reducidos a los asistentes como añadido a su ponencia. ComputerWorld y Universia colaboraron en calidad de media partners.

Este evento pudo llevarse a cabo gracias al comité organizador de la Universidad de Salamanca en conjunto con los miembros de RITSI y en colaboración con varios estudiantes de la Universidad de Salamanca, contando además con el apoyo institucional de este centro. Esta edición del Congreso RITSI ha sido la más multitudinaria hasta el momento, reuniendo a más de 900 estudiantes de Salamanca, Donostia-San Sebastian, Santander, Córdoba, Oviedo, Madrid, Murcia y otras universidades españolas, estudiantes de Formación Profesional y profesionales del sector, consiguiendo casi completar el aforo del Palacio de Congresos y superando así, nuevamente, todas las expectativas.

\section{IV Congreso Estatal RITSI}

La temática elegida para esta edición es "Seguridad en el ámbito de la Informática", tema siempre vigente en el sector, constituyendo una de las principales preocupaciones tanto de empresas como de usuarios, dado que la protección de la información confidencial que los usuarios confían en los sistemas informáticos de una empresa es siempre una cualidad que se debe garantizar.
 
\subsection{Objetivos}
Puesto que se trata de un evento al que se prevé la asistencia de estudiantes de todo el Estado, y teniendo en cuenta que nunca se ha celebrado un Congreso de la RITSI en la Comunidad Autónoma Vasca, la organización ha decidido enfocar el Congreso como un evento para la difusión del potencial industrial y tecnológico del País Vasco en general y de Donostia-San Sebastián en particular, en adición a los objetivos clásicos del evento. Se contemplan los siguientes objetivos:

\begin{itemize}
    \item Dar la oportunidad a los estudiantes de Ingeniería Informática para que comiencen la elaboración de sus redes de contactos con empresas y otros estudiantes, que serán muy útiles durante el desarrollo de su vida profesional.
    \item Fomentar las relaciones entre estudiantes de diferentes escuelas de Ingeniería Informática de España.
    \item Situar a RITSI como una fuente de apoyo y nexo para la comunicación entre los canales educativos y los canales profesionales, colaborando con las empresas para lograr la mejor colocación posible de los futuros titulados en Ingenierías en Informática, así como permitiendo a las mismas mostrar al alumnado sus metodologías de trabajo, productos, tecnologías, etcétera.
    \item Ser un escaparate en el cual mostrar el potencial industrial y tecnológico de la Comunidad Autónoma Vasca.
    \item Fomentar el interés por la actividad industrial y tecnológica en la Comunidad Universitaria en general, y en el alumnado y profesorado de los centros que se imparte Ingeniería Informática en la UPV/EHU en particular.
    \item Resaltar el papel de los futuros titulados como motor potencial para la economía local y el conocimiento universal.
    \item Ser fuente de acuerdos para la realización de prácticas en empresas y Proyectos de Fin de Carrera o Trabajos Fin de Grado.
\end{itemize}

\subsection{¿Dónde y cuándo?}

El IV Congreso tendrá lugar el día 15 de noviembre en la ciudad de Donostia. Galardonada con la Capitalía de la Cultura Europea para el año 2016, esta pequeña gran ciudad tiene una agenda cultural de primer nivel y puede considerarse la capital mundial del pintxo, siendo famosa su gastronomía.

El lugar elegido para celebrar el evento es la Palacio de Congresos Kursaal, una espectacular obra arquitectónica de Rafael Moneo frente al mar Cantábrico, epicentro de la actividad cultural y congresual de la ciudad, que acoge más de 300 eventos y 600.000 asistentes anuales. Para este Congreso se espera la máxima audiencia, suficiente para llenar su auditorio de más de 1100 plazas.

\subsection{Asistentes}

El evento está abierto a todo estudiante de Ingenierías Informáticas matriculado en las titulaciones existentes, así como a los representantes de estudiantes asistentes a la XXXIX Asamblea RITSI, además de a estudiantes de Formación Profesional en titulaciones informáticas y a profesionales del sector.

La tendencia a aumentar el número de asistentes es un gran indicador del interés que despierta este evento en el estudiantado, y se espera una gran presencia de alumnos tanto del País Vasco como de un gran número de universidades de toda la Península.

El entorno más cercano no solo incluye la Universidad del País Vasco. También la Universidad de Deusto y Mondragon Unibertsitatea tienen centros desde donde se esperan asistentes, además de todas las escuelas de Formación Profesional del entorno de Gipuzkoa y el País Vasco.

\noindent Están invitados a su vez al evento:

\begin{itemize}
    \item Juan Karlos Izagirre, alcalde de la ciudad, Donostia-San Sebastián.
    \item Iñaki Goirizelaia, Rector Magnífico de la Universidad del País Vasco / Euskal Herriko Unibertsitatea.
    \item Maite Zelaia, vicerrectora de Estudiantes, Empleo y Política Social de la UPV/EHU.
    \item Agustin Arruabarrena, decano de la Facultad de Informática de San Sebastián.
    \item Eduardo Vendrell, presidente de la Conferencia de Decanos y Directores de Ingeniería Informática.
    \item Tomas Iriondo, presidente de la asociación GAIA.
\end{itemize}

\section{Patrocinios y colaboración}

El patrocinio del IV Congreso RITSI es una excelente oportunidad para exponer su organización a todos los estudiantes asitentes, así como las instituciones y empresas que en ella participan, pudiéndoles mostrar su tecnología y servicios.

\paragraph{Patrocinio Platino} \EUR{3000}

\begin{itemize}
    \item La mención a la empresa como "Patrocinador Platino".
    \item Inclusión de ponencia de 45 minutos en el programa.
    \item Invitación a comida distendida con la Junta Directiva para discusión de futuras colaboraciones.
    \item Logo y mención destacada en todos los materiales publicitarios: acreditaciones, notas de prensa, página web, además de lugar destacado en cartel.
    \item Introducción de publicidad de la empresa o institución en la documentación del congreso.
\end{itemize}

\paragraph{Patrocinio Oro} \EUR{2000}

\begin{itemize}
    \item Inclusión de ponencia de 45 minutos en el programa.
    \item Logo y mención en todos los materiales publicitarios: acreditaciones, notas de prensa, página web, además de lugar destacado en cartel.
    \item Introducción de publicidad de la empresa o institución en la documentación del congreso.
\end{itemize}

\paragraph{Colaboración} \EUR{750}

\begin{itemize}
    \item Inclusión de logo en el cartel del Congreso.
    \item Inclusión de logo y mención en la web.
    \item Introducción de publicidad de la empresa o institución en la documentación del congreso.
\end{itemize}

Si alguno de los ponentes quisiera llevar a cabo un taller o colocar un stand publicitario durante el desarrollo, podrá contactar con nosotros directamente. La disponibilidad estará sujeta la del espacio y al programa del evento.

Los precios anteriores no incluyen el Impuesto sobre el Valor Añadido (IVA).

\section{Nuestra experiencia}
El éxito de las ediciones anteriores prueba el compromiso de RITSI con este proyecto, contando además con la experiencia que estas han otorgado. La Asociación, que cree firmemente en el Congreso como herramienta de difusión y promoción de nuevas tecnologías, novedades en el sector o metodologías de trabajo, además de nexo entre empresas y alumnos, pudiéndose dar a conocer los intereses de unos y otros de una forma directa, apuesta por esta edición con la misma dedicación que en las ediciones precedentes.

\section{Comité Organizador y contacto}
Si desea más información sobre la asociación o las anteriores ediciones del congreso, puede visitar la página web de la asociación:

\indent Link: \href{http://www.ritsi.org}{ritsi.org}

En cambio, si lo que desea es más información sobre el IV Congreso Estatal RITSI o patrocinar el mismo, puede contactar directamente con el Comité Organizador:

\begin{quotation}
\noindent A la atención de \textbf{Gorka Maiztegi Etxeberria} \\
Email: \href{mailto:congreso4@ritsi.org}{congreso4@ritsi.org} \\
Teléfono: \href{tel:658714863}{658714863}
\end{quotation}

\end{document}
